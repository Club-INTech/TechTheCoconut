% Generated by Sphinx.
\def\sphinxdocclass{report}
\documentclass[letterpaper,10pt,french]{sphinxmanual}
\usepackage[utf8]{inputenc}
\DeclareUnicodeCharacter{00A0}{\nobreakspace}
\usepackage[T1]{fontenc}
\usepackage{babel}
\usepackage{times}
\usepackage[Sonny]{fncychap}
\usepackage{longtable}
\usepackage{sphinx}
\usepackage{multirow}


\title{Tech The Coconut Documentation}
\date{03 January 2012}
\release{0.0.1}
\author{Club INTech}
\newcommand{\sphinxlogo}{}
\renewcommand{\releasename}{Version}
\makeindex

\makeatletter
\def\PYG@reset{\let\PYG@it=\relax \let\PYG@bf=\relax%
    \let\PYG@ul=\relax \let\PYG@tc=\relax%
    \let\PYG@bc=\relax \let\PYG@ff=\relax}
\def\PYG@tok#1{\csname PYG@tok@#1\endcsname}
\def\PYG@toks#1+{\ifx\relax#1\empty\else%
    \PYG@tok{#1}\expandafter\PYG@toks\fi}
\def\PYG@do#1{\PYG@bc{\PYG@tc{\PYG@ul{%
    \PYG@it{\PYG@bf{\PYG@ff{#1}}}}}}}
\def\PYG#1#2{\PYG@reset\PYG@toks#1+\relax+\PYG@do{#2}}

\def\PYG@tok@gd{\def\PYG@tc##1{\textcolor[rgb]{0.63,0.00,0.00}{##1}}}
\def\PYG@tok@gu{\let\PYG@bf=\textbf\def\PYG@tc##1{\textcolor[rgb]{0.50,0.00,0.50}{##1}}}
\def\PYG@tok@gt{\def\PYG@tc##1{\textcolor[rgb]{0.00,0.25,0.82}{##1}}}
\def\PYG@tok@gs{\let\PYG@bf=\textbf}
\def\PYG@tok@gr{\def\PYG@tc##1{\textcolor[rgb]{1.00,0.00,0.00}{##1}}}
\def\PYG@tok@cm{\let\PYG@it=\textit\def\PYG@tc##1{\textcolor[rgb]{0.25,0.50,0.56}{##1}}}
\def\PYG@tok@vg{\def\PYG@tc##1{\textcolor[rgb]{0.73,0.38,0.84}{##1}}}
\def\PYG@tok@m{\def\PYG@tc##1{\textcolor[rgb]{0.13,0.50,0.31}{##1}}}
\def\PYG@tok@mh{\def\PYG@tc##1{\textcolor[rgb]{0.13,0.50,0.31}{##1}}}
\def\PYG@tok@cs{\def\PYG@tc##1{\textcolor[rgb]{0.25,0.50,0.56}{##1}}\def\PYG@bc##1{\colorbox[rgb]{1.00,0.94,0.94}{##1}}}
\def\PYG@tok@ge{\let\PYG@it=\textit}
\def\PYG@tok@vc{\def\PYG@tc##1{\textcolor[rgb]{0.73,0.38,0.84}{##1}}}
\def\PYG@tok@il{\def\PYG@tc##1{\textcolor[rgb]{0.13,0.50,0.31}{##1}}}
\def\PYG@tok@go{\def\PYG@tc##1{\textcolor[rgb]{0.19,0.19,0.19}{##1}}}
\def\PYG@tok@cp{\def\PYG@tc##1{\textcolor[rgb]{0.00,0.44,0.13}{##1}}}
\def\PYG@tok@gi{\def\PYG@tc##1{\textcolor[rgb]{0.00,0.63,0.00}{##1}}}
\def\PYG@tok@gh{\let\PYG@bf=\textbf\def\PYG@tc##1{\textcolor[rgb]{0.00,0.00,0.50}{##1}}}
\def\PYG@tok@ni{\let\PYG@bf=\textbf\def\PYG@tc##1{\textcolor[rgb]{0.84,0.33,0.22}{##1}}}
\def\PYG@tok@nl{\let\PYG@bf=\textbf\def\PYG@tc##1{\textcolor[rgb]{0.00,0.13,0.44}{##1}}}
\def\PYG@tok@nn{\let\PYG@bf=\textbf\def\PYG@tc##1{\textcolor[rgb]{0.05,0.52,0.71}{##1}}}
\def\PYG@tok@no{\def\PYG@tc##1{\textcolor[rgb]{0.38,0.68,0.84}{##1}}}
\def\PYG@tok@na{\def\PYG@tc##1{\textcolor[rgb]{0.25,0.44,0.63}{##1}}}
\def\PYG@tok@nb{\def\PYG@tc##1{\textcolor[rgb]{0.00,0.44,0.13}{##1}}}
\def\PYG@tok@nc{\let\PYG@bf=\textbf\def\PYG@tc##1{\textcolor[rgb]{0.05,0.52,0.71}{##1}}}
\def\PYG@tok@nd{\let\PYG@bf=\textbf\def\PYG@tc##1{\textcolor[rgb]{0.33,0.33,0.33}{##1}}}
\def\PYG@tok@ne{\def\PYG@tc##1{\textcolor[rgb]{0.00,0.44,0.13}{##1}}}
\def\PYG@tok@nf{\def\PYG@tc##1{\textcolor[rgb]{0.02,0.16,0.49}{##1}}}
\def\PYG@tok@si{\let\PYG@it=\textit\def\PYG@tc##1{\textcolor[rgb]{0.44,0.63,0.82}{##1}}}
\def\PYG@tok@s2{\def\PYG@tc##1{\textcolor[rgb]{0.25,0.44,0.63}{##1}}}
\def\PYG@tok@vi{\def\PYG@tc##1{\textcolor[rgb]{0.73,0.38,0.84}{##1}}}
\def\PYG@tok@nt{\let\PYG@bf=\textbf\def\PYG@tc##1{\textcolor[rgb]{0.02,0.16,0.45}{##1}}}
\def\PYG@tok@nv{\def\PYG@tc##1{\textcolor[rgb]{0.73,0.38,0.84}{##1}}}
\def\PYG@tok@s1{\def\PYG@tc##1{\textcolor[rgb]{0.25,0.44,0.63}{##1}}}
\def\PYG@tok@gp{\let\PYG@bf=\textbf\def\PYG@tc##1{\textcolor[rgb]{0.78,0.36,0.04}{##1}}}
\def\PYG@tok@sh{\def\PYG@tc##1{\textcolor[rgb]{0.25,0.44,0.63}{##1}}}
\def\PYG@tok@ow{\let\PYG@bf=\textbf\def\PYG@tc##1{\textcolor[rgb]{0.00,0.44,0.13}{##1}}}
\def\PYG@tok@sx{\def\PYG@tc##1{\textcolor[rgb]{0.78,0.36,0.04}{##1}}}
\def\PYG@tok@bp{\def\PYG@tc##1{\textcolor[rgb]{0.00,0.44,0.13}{##1}}}
\def\PYG@tok@c1{\let\PYG@it=\textit\def\PYG@tc##1{\textcolor[rgb]{0.25,0.50,0.56}{##1}}}
\def\PYG@tok@kc{\let\PYG@bf=\textbf\def\PYG@tc##1{\textcolor[rgb]{0.00,0.44,0.13}{##1}}}
\def\PYG@tok@c{\let\PYG@it=\textit\def\PYG@tc##1{\textcolor[rgb]{0.25,0.50,0.56}{##1}}}
\def\PYG@tok@mf{\def\PYG@tc##1{\textcolor[rgb]{0.13,0.50,0.31}{##1}}}
\def\PYG@tok@err{\def\PYG@bc##1{\fcolorbox[rgb]{1.00,0.00,0.00}{1,1,1}{##1}}}
\def\PYG@tok@kd{\let\PYG@bf=\textbf\def\PYG@tc##1{\textcolor[rgb]{0.00,0.44,0.13}{##1}}}
\def\PYG@tok@ss{\def\PYG@tc##1{\textcolor[rgb]{0.32,0.47,0.09}{##1}}}
\def\PYG@tok@sr{\def\PYG@tc##1{\textcolor[rgb]{0.14,0.33,0.53}{##1}}}
\def\PYG@tok@mo{\def\PYG@tc##1{\textcolor[rgb]{0.13,0.50,0.31}{##1}}}
\def\PYG@tok@mi{\def\PYG@tc##1{\textcolor[rgb]{0.13,0.50,0.31}{##1}}}
\def\PYG@tok@kn{\let\PYG@bf=\textbf\def\PYG@tc##1{\textcolor[rgb]{0.00,0.44,0.13}{##1}}}
\def\PYG@tok@o{\def\PYG@tc##1{\textcolor[rgb]{0.40,0.40,0.40}{##1}}}
\def\PYG@tok@kr{\let\PYG@bf=\textbf\def\PYG@tc##1{\textcolor[rgb]{0.00,0.44,0.13}{##1}}}
\def\PYG@tok@s{\def\PYG@tc##1{\textcolor[rgb]{0.25,0.44,0.63}{##1}}}
\def\PYG@tok@kp{\def\PYG@tc##1{\textcolor[rgb]{0.00,0.44,0.13}{##1}}}
\def\PYG@tok@w{\def\PYG@tc##1{\textcolor[rgb]{0.73,0.73,0.73}{##1}}}
\def\PYG@tok@kt{\def\PYG@tc##1{\textcolor[rgb]{0.56,0.13,0.00}{##1}}}
\def\PYG@tok@sc{\def\PYG@tc##1{\textcolor[rgb]{0.25,0.44,0.63}{##1}}}
\def\PYG@tok@sb{\def\PYG@tc##1{\textcolor[rgb]{0.25,0.44,0.63}{##1}}}
\def\PYG@tok@k{\let\PYG@bf=\textbf\def\PYG@tc##1{\textcolor[rgb]{0.00,0.44,0.13}{##1}}}
\def\PYG@tok@se{\let\PYG@bf=\textbf\def\PYG@tc##1{\textcolor[rgb]{0.25,0.44,0.63}{##1}}}
\def\PYG@tok@sd{\let\PYG@it=\textit\def\PYG@tc##1{\textcolor[rgb]{0.25,0.44,0.63}{##1}}}

\def\PYGZbs{\char`\\}
\def\PYGZus{\char`\_}
\def\PYGZob{\char`\{}
\def\PYGZcb{\char`\}}
\def\PYGZca{\char`\^}
\def\PYGZsh{\char`\#}
\def\PYGZpc{\char`\%}
\def\PYGZdl{\char`\$}
\def\PYGZti{\char`\~}
% for compatibility with earlier versions
\def\PYGZat{@}
\def\PYGZlb{[}
\def\PYGZrb{]}
\makeatother

\begin{document}

\maketitle
\tableofcontents
\phantomsection\label{index::doc}


Contents:


\chapter{Actionneur}
\label{actionneur:actionneur}\label{actionneur::doc}\label{actionneur:welcome-to-tech-the-coconut-s-documentation}\label{actionneur:module-lib.actionneur}\index{lib.actionneur (module)}\index{Actionneur (classe dans lib.actionneur)}

\begin{fulllineitems}
\phantomsection\label{actionneur:lib.actionneur.Actionneur}\pysigline{\strong{class }\code{lib.actionneur.}\bfcode{Actionneur}}
Classe permettant de gérer un actionneur

\end{fulllineitems}



\chapter{Actions}
\label{actions::doc}\label{actions:actions}

\section{Actions\_in}
\label{actions:actions-in}
Actions reçues


\subsection{Actionneurs}
\label{actions:actionneurs}\label{actions:module-lib.actions_in.actionneurs_in}\index{lib.actions\_in.actionneurs\_in (module)}\index{Actionneurs\_in (classe dans lib.actions\_in.actionneurs\_in)}

\begin{fulllineitems}
\phantomsection\label{actions:lib.actions_in.actionneurs_in.Actionneurs_in}\pysigline{\strong{class }\code{lib.actions\_in.actionneurs\_in.}\bfcode{Actionneurs\_in}}
Classe permettant de gérer les informations envoyées par les actionneurs

\end{fulllineitems}



\subsection{Asservissement}
\label{actions:module-lib.actions_in.asservissement_in}\label{actions:asservissement}\index{lib.actions\_in.asservissement\_in (module)}\index{Asservissement\_in (classe dans lib.actions\_in.asservissement\_in)}

\begin{fulllineitems}
\phantomsection\label{actions:lib.actions_in.asservissement_in.Asservissement_in}\pysigline{\strong{class }\code{lib.actions\_in.asservissement\_in.}\bfcode{Asservissement\_in}}
Classe permettant de gérer les informations envoyées par la carte d'asservissement
\begin{quote}\begin{description}
\item[{Todo }] \leavevmode
Créer une liaison série dans le constructeur si ce n'est pas déjà fait

\item[{Todo }] \leavevmode
Supprimer l'objet série dans le destructeur

\end{description}\end{quote}

\end{fulllineitems}



\subsection{Capteurs}
\label{actions:capteurs}\label{actions:module-lib.actions_in.capteurs_in}\index{lib.actions\_in.capteurs\_in (module)}\index{Capteurs\_in (classe dans lib.actions\_in.capteurs\_in)}

\begin{fulllineitems}
\phantomsection\label{actions:lib.actions_in.capteurs_in.Capteurs_in}\pysigline{\strong{class }\code{lib.actions\_in.capteurs\_in.}\bfcode{Capteurs\_in}}
Classe permettant de gérer les informations envoyées par les capteurs

\end{fulllineitems}



\section{Actions\_out}
\label{actions:actions-out}
Actions envoyées


\subsection{Actionneurs}
\label{actions:id1}\phantomsection\label{actions:module-lib.actions_out.actionneurs_out}\index{lib.actions\_out.actionneurs\_out (module)}\index{Actionneurs\_out (classe dans lib.actions\_out.actionneurs\_out)}

\begin{fulllineitems}
\phantomsection\label{actions:lib.actions_out.actionneurs_out.Actionneurs_out}\pysigline{\strong{class }\code{lib.actions\_out.actionneurs\_out.}\bfcode{Actionneurs\_out}}
Classe permettant de gérer les informations à envoyer aux actionneurs

\end{fulllineitems}



\subsection{Asservissement}
\label{actions:id2}\phantomsection\label{actions:module-lib.actions_out.asservissement_out}\index{lib.actions\_out.asservissement\_out (module)}\index{Asservissement\_out (classe dans lib.actions\_out.asservissement\_out)}

\begin{fulllineitems}
\phantomsection\label{actions:lib.actions_out.asservissement_out.Asservissement_out}\pysigline{\strong{class }\code{lib.actions\_out.asservissement\_out.}\bfcode{Asservissement\_out}}
Classe permettant de gérer les informations à envoyer à l'asservissement
\begin{quote}\begin{description}
\item[{Todo }] \leavevmode
Créer une liaison série dans le constructeur si ce n'est pas déjà fait

\item[{Todo }] \leavevmode
Supprimer l'objet série dans le destructeur

\end{description}\end{quote}

\end{fulllineitems}



\subsection{Capteurs}
\label{actions:id3}\phantomsection\label{actions:module-lib.actions_out.capteurs_out}\index{lib.actions\_out.capteurs\_out (module)}\index{Capteurs\_out (classe dans lib.actions\_out.capteurs\_out)}

\begin{fulllineitems}
\phantomsection\label{actions:lib.actions_out.capteurs_out.Capteurs_out}\pysigline{\strong{class }\code{lib.actions\_out.capteurs\_out.}\bfcode{Capteurs\_out}}
Classe permettant de gérer les informations à envoyer aux capteurs

\end{fulllineitems}



\chapter{Asservissement}
\label{asservissement:asservissement}\label{asservissement::doc}\label{asservissement:module-lib.asservissement}\index{lib.asservissement (module)}\index{Asservissement (classe dans lib.asservissement)}

\begin{fulllineitems}
\phantomsection\label{asservissement:lib.asservissement.Asservissement}\pysigline{\strong{class }\code{lib.asservissement.}\bfcode{Asservissement}}
Classe de gérer l'asservissement

\end{fulllineitems}



\chapter{Capteur}
\label{capteur:module-lib.capteur}\label{capteur::doc}\label{capteur:capteur}\index{lib.capteur (module)}\index{Capteur (classe dans lib.capteur)}

\begin{fulllineitems}
\phantomsection\label{capteur:lib.capteur.Capteur}\pysigline{\strong{class }\code{lib.capteur.}\bfcode{Capteur}}
Classe permettant de gérer un capteur

\end{fulllineitems}



\chapter{Carte}
\label{carte:module-lib.carte}\label{carte::doc}\label{carte:carte}\index{lib.carte (module)}\index{Carte (classe dans lib.carte)}

\begin{fulllineitems}
\phantomsection\label{carte:lib.carte.Carte}\pysigline{\strong{class }\code{lib.carte.}\bfcode{Carte}}
Classe permettant de gérer l'aire de jeu

\end{fulllineitems}



\chapter{Décision}
\label{decision:module-lib.decision}\label{decision:decision}\label{decision::doc}\index{lib.decision (module)}\index{Decision (classe dans lib.decision)}

\begin{fulllineitems}
\phantomsection\label{decision:lib.decision.Decision}\pysigline{\strong{class }\code{lib.decision.}\bfcode{Decision}}
Classe permettant de prendre une décision stratégique

\end{fulllineitems}



\chapter{Éléments de jeu}
\label{elements_jeu:elements-de-jeu}\label{elements_jeu::doc}\label{elements_jeu:module-lib.elements_jeu}\index{lib.elements\_jeu (module)}\index{Carte\_tresor (classe dans lib.elements\_jeu)}

\begin{fulllineitems}
\phantomsection\label{elements_jeu:lib.elements_jeu.Carte_tresor}\pysigline{\strong{class }\code{lib.elements\_jeu.}\bfcode{Carte\_tresor}}
Classe de créer l'élément de jeu carte au trésor

\end{fulllineitems}

\index{Disque (classe dans lib.elements\_jeu)}

\begin{fulllineitems}
\phantomsection\label{elements_jeu:lib.elements_jeu.Disque}\pysigline{\strong{class }\code{lib.elements\_jeu.}\bfcode{Disque}}
Classe de créer l'élément de jeu disque

\end{fulllineitems}

\index{Lingot (classe dans lib.elements\_jeu)}

\begin{fulllineitems}
\phantomsection\label{elements_jeu:lib.elements_jeu.Lingot}\pysigline{\strong{class }\code{lib.elements\_jeu.}\bfcode{Lingot}}
Classe de créer l'élément de jeu lingot

\end{fulllineitems}

\index{Poussoir (classe dans lib.elements\_jeu)}

\begin{fulllineitems}
\phantomsection\label{elements_jeu:lib.elements_jeu.Poussoir}\pysigline{\strong{class }\code{lib.elements\_jeu.}\bfcode{Poussoir}}
Classe de créer l'élément de jeu poussoir

\end{fulllineitems}

\index{Totem (classe dans lib.elements\_jeu)}

\begin{fulllineitems}
\phantomsection\label{elements_jeu:lib.elements_jeu.Totem}\pysigline{\strong{class }\code{lib.elements\_jeu.}\bfcode{Totem}}
Classe de créer l'élément de jeu totem

\end{fulllineitems}

\index{Zone (classe dans lib.elements\_jeu)}

\begin{fulllineitems}
\phantomsection\label{elements_jeu:lib.elements_jeu.Zone}\pysigline{\strong{class }\code{lib.elements\_jeu.}\bfcode{Zone}}
Classe de créer l'élément de jeu zone
\begin{quote}\begin{description}
\item[{Todo }] \leavevmode
Différencier les différents types de zone (départ, cale, pont, ...) avec leurs propriétés

\end{description}\end{quote}

\end{fulllineitems}



\chapter{Évènement}
\label{evenement:evenement}\label{evenement:module-lib.evenement}\label{evenement::doc}\index{lib.evenement (module)}\index{Evenement (classe dans lib.evenement)}

\begin{fulllineitems}
\phantomsection\label{evenement:lib.evenement.Evenement}\pysigline{\strong{class }\code{lib.evenement.}\bfcode{Evenement}}
Classe permettant de gérer un événement, par exemple pour savoir si l'évitement doit le prendre en compte.

\end{fulllineitems}



\chapter{I2c}
\label{i2c:i2c}\label{i2c::doc}\label{i2c:module-lib.i2c}\index{lib.i2c (module)}\index{I2c (classe dans lib.i2c)}

\begin{fulllineitems}
\phantomsection\label{i2c:lib.i2c.I2c}\pysigline{\strong{class }\code{lib.i2c.}\bfcode{I2c}}
Classe de créer une liaison I2c
\begin{quote}\begin{description}
\item[{Todo }] \leavevmode
Trouver une librairie pour gérer l'i2c

\item[{Todo }] \leavevmode
Initialiser la liaison i2c dans le constructeur

\item[{Todo }] \leavevmode
Fermer proprement la liaison i2c dans le destructeur

\end{description}\end{quote}

\end{fulllineitems}



\chapter{Jeu}
\label{jeu:module-lib.jeu}\label{jeu::doc}\label{jeu:jeu}\index{lib.jeu (module)}\index{Jeu (classe dans lib.jeu)}

\begin{fulllineitems}
\phantomsection\label{jeu:lib.jeu.Jeu}\pysigline{\strong{class }\code{lib.jeu.}\bfcode{Jeu}}
Classe de gérer le déroulement d'une partie

\end{fulllineitems}



\chapter{Log}
\label{log:module-lib.log}\label{log::doc}\label{log:log}\index{lib.log (module)}\index{Log (classe dans lib.log)}

\begin{fulllineitems}
\phantomsection\label{log:lib.log.Log}\pysiglinewithargsret{\strong{class }\code{lib.log.}\bfcode{Log}}{\emph{logs=None}, \emph{logs\_level=None}, \emph{logs\_format=None}, \emph{stderr=None}, \emph{stderr\_level=None}, \emph{stderr\_format=None}, \emph{dossier=None}}{}
Classe permettant de gérer les logs

Pour utiliser les logs dans vos fichiers (sauf dans le lanceur où il faut préciser les paramètres) :

Pour charger le système de log correctement, en début de fichier ajoutez :

import lib.log

log = lib.log.Log()

Puis vous pouvez logguer des messages avec (dans ordre croissant de niveau) :

log.logger.debug(`mon message')

log.logger.info(`mon message')

log.logger.warning(`mon message')

log.logger.error(`mon message')

log.logger.critical(`mon message')

L'arborescence des fichiers de logs est : {[}annee{]}-{[}mois{]}-{[}jour{]}/{[}revision{]}.log
\begin{quote}\begin{description}
\item[{Paramètres}] \leavevmode\begin{itemize}
\item {} 
\textbf{logs} (\href{http://docs.python.org/library/functions.html\#bool}{\emph{bool}}) -- Enregistrer dans les fichiers de log ?

\item {} 
\textbf{logs\_level} (\emph{string `DEBUG'\textbar{}'INFO'\textbar{}'WARNING'\textbar{}'ERROR'\textbar{}'CRITICAL'}) -- Enregistrer à partir de quel niveau de log ?

\item {} 
\textbf{logs\_format} (\href{http://docs.python.org/library/string.html\#module-string}{\emph{string}}) -- Format des logs (voir \href{http://docs.python.org/library/logging.html\#logrecord-attributes}{http://docs.python.org/library/logging.html\#logrecord-attributes}). Ex : `\%(asctime)s:\%(name)s:\%(levelname)s:\%(message)s'

\item {} 
\textbf{stderr} (\href{http://docs.python.org/library/functions.html\#bool}{\emph{bool}}) -- Afficher les erreur dans le stderr ? (ie à l'écran)

\item {} 
\textbf{stderr\_level} (\emph{string `DEBUG'\textbar{}'INFO'\textbar{}'WARNING'\textbar{}'ERROR'\textbar{}'CRITICAL'}) -- Afficher sur l'écran à partir de quel niveau de log ?

\item {} 
\textbf{stderr\_format} (\href{http://docs.python.org/library/string.html\#module-string}{\emph{string}}) -- Format d'affiche à l'écran (voir \href{http://docs.python.org/library/logging.html\#logrecord-attributes}{http://docs.python.org/library/logging.html\#logrecord-attributes}). Ex : `\%(asctime)s:\%(name)s:\%(levelname)s:\%(message)s'

\item {} 
\textbf{dossier} (\href{http://docs.python.org/library/string.html\#module-string}{\emph{string}}) -- Dossier où mettre les logs (à partir de la racine du code, c'est-à-dire le dossier contenant lanceur.py). Ex : `logs'

\end{itemize}

\item[{Todo }] \leavevmode
Mettre les valeurs qui vont bien dans les profils de configuration

\end{description}\end{quote}
\index{configurer\_logs() (méthode lib.log.Log)}

\begin{fulllineitems}
\phantomsection\label{log:lib.log.Log.configurer_logs}\pysiglinewithargsret{\bfcode{configurer\_logs}}{}{}
Configure les logs (handler)

\end{fulllineitems}

\index{configurer\_stderr() (méthode lib.log.Log)}

\begin{fulllineitems}
\phantomsection\label{log:lib.log.Log.configurer_stderr}\pysiglinewithargsret{\bfcode{configurer\_stderr}}{}{}
Configure la sortie stderr (handler)

\end{fulllineitems}

\index{creer\_dossier() (méthode lib.log.Log)}

\begin{fulllineitems}
\phantomsection\label{log:lib.log.Log.creer_dossier}\pysiglinewithargsret{\bfcode{creer\_dossier}}{\emph{dossier}}{}
Crée un dossier si il n'existe pas déjà
\begin{quote}\begin{description}
\item[{Paramètres}] \leavevmode
\textbf{dossier} (\href{http://docs.python.org/library/string.html\#module-string}{\emph{string}}) -- chemin vers le dossier à créer

\item[{Retourne}] \leavevmode
True si on a eu besoin de créer le dossier, False si il existait déjà

\item[{Type retourné}] \leavevmode
bool

\end{description}\end{quote}

\end{fulllineitems}

\index{ecrire\_entete() (méthode lib.log.Log)}

\begin{fulllineitems}
\phantomsection\label{log:lib.log.Log.ecrire_entete}\pysiglinewithargsret{\bfcode{ecrire\_entete}}{}{}
Crée l'entête dans les logs au niveau INFO

\end{fulllineitems}

\index{initialisation() (méthode lib.log.Log)}

\begin{fulllineitems}
\phantomsection\label{log:lib.log.Log.initialisation}\pysiglinewithargsret{\bfcode{initialisation}}{\emph{logs}, \emph{logs\_level}, \emph{logs\_format}, \emph{stderr}, \emph{stderr\_level}, \emph{stderr\_format}, \emph{dossier}}{}
Initialise le système de log
\begin{quote}\begin{description}
\item[{Paramètres}] \leavevmode\begin{itemize}
\item {} 
\textbf{logs} (\href{http://docs.python.org/library/functions.html\#bool}{\emph{bool}}) -- Enregistrer dans les fichiers de log ?

\item {} 
\textbf{logs\_level} (\emph{string `DEBUG'\textbar{}'INFO'\textbar{}'WARNING'\textbar{}'ERROR'\textbar{}'CRITICAL'}) -- Enregistrer à partir de quel niveau de log ?

\item {} 
\textbf{logs\_format} (\href{http://docs.python.org/library/string.html\#module-string}{\emph{string}}) -- Format des logs (voir \href{http://docs.python.org/library/logging.html\#logrecord-attributes}{http://docs.python.org/library/logging.html\#logrecord-attributes}). Ex : `\%(asctime)s:\%(name)s:\%(levelname)s:\%(message)s'

\item {} 
\textbf{stderr} (\href{http://docs.python.org/library/functions.html\#bool}{\emph{bool}}) -- Afficher les erreur dans le stderr ? (ie à l'écran)

\item {} 
\textbf{stderr\_level} (\emph{string `DEBUG'\textbar{}'INFO'\textbar{}'WARNING'\textbar{}'ERROR'\textbar{}'CRITICAL'}) -- Afficher sur l'écran à partir de quel niveau de log ?

\item {} 
\textbf{stderr\_format} (\href{http://docs.python.org/library/string.html\#module-string}{\emph{string}}) -- Format d'affiche à l'écran (voir \href{http://docs.python.org/library/logging.html\#logrecord-attributes}{http://docs.python.org/library/logging.html\#logrecord-attributes}). Ex : `\%(asctime)s:\%(name)s:\%(levelname)s:\%(message)s'

\item {} 
\textbf{dossier} (\href{http://docs.python.org/library/string.html\#module-string}{\emph{string}}) -- Dossier où mettre les logs (à partir de la racine du code, c'est-à-dire le dossier contenant lanceur.py). Ex : `logs'

\end{itemize}

\item[{Retourne}] \leavevmode
Statut de l'initialisation. True si réussite, False si échec

\item[{Type retourné}] \leavevmode
bool

\end{description}\end{quote}

\end{fulllineitems}

\index{revision\_disponible() (méthode lib.log.Log)}

\begin{fulllineitems}
\phantomsection\label{log:lib.log.Log.revision_disponible}\pysiglinewithargsret{\bfcode{revision\_disponible}}{\emph{dossier}, \emph{dossier\_date}}{}
Donne la prochaine révision à créer dans les logs
\begin{quote}\begin{description}
\item[{Paramètres}] \leavevmode\begin{itemize}
\item {} 
\textbf{dossier} (\href{http://docs.python.org/library/string.html\#module-string}{\emph{string}}) -- dossier principal des logs

\item {} 
\textbf{dossier\_date} (\href{http://docs.python.org/library/string.html\#module-string}{\emph{string}}) -- dossier de la date actuelle

\end{itemize}

\item[{Retourne}] \leavevmode
révision à créer

\item[{Type retourné}] \leavevmode
int

\end{description}\end{quote}

\end{fulllineitems}


\end{fulllineitems}



\chapter{Math}
\label{math::doc}\label{math:math}

\section{Point}
\label{math:point}\label{math:module-lib.math.point}\index{lib.math.point (module)}\index{Point (classe dans lib.math.point)}

\begin{fulllineitems}
\phantomsection\label{math:lib.math.point.Point}\pysiglinewithargsret{\strong{class }\code{lib.math.point.}\bfcode{Point}}{\emph{x}, \emph{y}}{}
Classe permettant de définir un point mathématique dans R\textasciicircum{}2 et les opérations usuelles sur les points dans R\textasciicircum{}2
\begin{quote}\begin{description}
\item[{Paramètres}] \leavevmode\begin{itemize}
\item {} 
\textbf{x} (\href{http://docs.python.org/library/functions.html\#float}{\emph{float}}) -- abscisse

\item {} 
\textbf{y} (\href{http://docs.python.org/library/functions.html\#float}{\emph{float}}) -- ordonnée

\end{itemize}

\end{description}\end{quote}

\end{fulllineitems}



\section{Vecteur}
\label{math:module-lib.math.vecteur}\label{math:vecteur}\index{lib.math.vecteur (module)}\index{Vecteur (classe dans lib.math.vecteur)}

\begin{fulllineitems}
\phantomsection\label{math:lib.math.vecteur.Vecteur}\pysiglinewithargsret{\strong{class }\code{lib.math.vecteur.}\bfcode{Vecteur}}{\emph{a}, \emph{b}}{}
Classe permettant de définir un vecteur mathématique dans R\textasciicircum{}2 et les opérations usuelles sur les vecteurs dans R\textasciicircum{}2
\begin{quote}\begin{description}
\item[{Paramètres}] \leavevmode\begin{itemize}
\item {} 
\textbf{a} (\emph{Point}) -- 1er point

\item {} 
\textbf{b} (\emph{Point}) -- 2ème point

\end{itemize}

\end{description}\end{quote}
\index{angle() (méthode lib.math.vecteur.Vecteur)}

\begin{fulllineitems}
\phantomsection\label{math:lib.math.vecteur.Vecteur.angle}\pysiglinewithargsret{\bfcode{angle}}{}{}~\begin{quote}\begin{description}
\item[{Type retourné}] \leavevmode
float

\item[{Retourne}] \leavevmode
angle orienté en radians entre le vecteur et l'axe des abscisses

\end{description}\end{quote}

\end{fulllineitems}

\index{norme() (méthode lib.math.vecteur.Vecteur)}

\begin{fulllineitems}
\phantomsection\label{math:lib.math.vecteur.Vecteur.norme}\pysiglinewithargsret{\bfcode{norme}}{}{}~\begin{quote}\begin{description}
\item[{Type retourné}] \leavevmode
float

\item[{Retourne}] \leavevmode
norme euclidienne du vecteur

\end{description}\end{quote}

\end{fulllineitems}


\end{fulllineitems}



\chapter{Périphérique}
\label{peripherique:module-lib.peripherique}\label{peripherique::doc}\label{peripherique:peripherique}\index{lib.peripherique (module)}\index{Peripherique (classe dans lib.peripherique)}

\begin{fulllineitems}
\phantomsection\label{peripherique:lib.peripherique.Peripherique}\pysigline{\strong{class }\code{lib.peripherique.}\bfcode{Peripherique}}
Classe de gérer un périphérique
\begin{quote}\begin{description}
\item[{Todo }] \leavevmode
Permettre la découverte de périphériques

\end{description}\end{quote}

\end{fulllineitems}



\chapter{Conf}
\label{profils::doc}\label{profils:conf}

\section{Profils}
\label{profils:profils}

\subsection{Constantes}
\label{profils:constantes}

\subsubsection{develop}
\label{profils:develop}\label{profils:module-profils.develop.constantes}\index{profils.develop.constantes (module)}\begin{itemize}
\item {} 
constantes{[}''Coconut''{]}{[}''largeur''{]} Largeur en mm

\item {} 
constantes{[}''Coconut''{]}{[}''longueur''{]} Longueur en mm

\end{itemize}
\index{constantes (attribut profils.develop.constantes)}

\begin{fulllineitems}
\phantomsection\label{profils:profils.develop.constantes.constantes}\pysigline{\code{constantes.}\bfcode{constantes}\strong{ = \{`Table': \{\}, `Coconut': \{`longueur': 350, `largeur': 150\}, `Logs': \{`logs\_level': `DEBUG', `logs': True, `logs\_format': `\%(asctime)s:\%(name)s:\%(levelname)s:\%(threadName)s:l\%(lineno)d:\%(message)s', `dossier': `logs', `stderr\_format': `\%(asctime)s:\%(name)s:\%(levelname)s:\%(threadName)s:l\%(lineno)d:\%(message)s', `stderr': True, `stderr\_level': `DEBUG'\}, `Anna': \{\}\}}}
\end{fulllineitems}



\subsubsection{developSimulUc}
\label{profils:developsimuluc}\label{profils:module-profils.developSimulUc.constantes}\index{profils.developSimulUc.constantes (module)}\begin{itemize}
\item {} 
constantes{[}''Coconut''{]}{[}''largeur''{]} Largeur en mm

\item {} 
constantes{[}''Coconut''{]}{[}''longueur''{]} Longueur en mm

\end{itemize}
\index{constantes (attribut profils.developSimulUc.constantes)}

\begin{fulllineitems}
\phantomsection\label{profils:profils.developSimulUc.constantes.constantes}\pysigline{\code{constantes.}\bfcode{constantes}\strong{ = \{`Table': \{\}, `Coconut': \{`longueur': 350, `largeur': 150\}, `Logs': \{`logs\_level': `DEBUG', `logs': True, `logs\_format': `\%(asctime)s:\%(name)s:\%(levelname)s:\%(threadName)s:l\%(lineno)d:\%(message)s', `dossier': `logs', `stderr\_format': `\%(asctime)s:\%(name)s:\%(levelname)s:\%(threadName)s:l\%(lineno)d:\%(message)s', `stderr': True, `stderr\_level': `DEBUG'\}, `Anna': \{\}\}}}
\end{fulllineitems}



\subsubsection{prod}
\label{profils:prod}\label{profils:module-profils.prod.constantes}\index{profils.prod.constantes (module)}\begin{itemize}
\item {} 
constantes{[}''Coconut''{]}{[}''largeur''{]} Largeur en mm

\item {} 
constantes{[}''Coconut''{]}{[}''longueur''{]} Longueur en mm

\end{itemize}
\index{constantes (attribut profils.prod.constantes)}

\begin{fulllineitems}
\phantomsection\label{profils:profils.prod.constantes.constantes}\pysigline{\code{constantes.}\bfcode{constantes}\strong{ = \{`Table': \{\}, `Coconut': \{`longueur': 350, `largeur': 150\}, `Logs': \{`logs\_level': `DEBUG', `logs': True, `logs\_format': `\%(asctime)s:\%(name)s:\%(levelname)s:\%(threadName)s:l\%(lineno)d:\%(message)s', `dossier': `logs', `stderr\_format': `\%(asctime)s:\%(name)s:\%(levelname)s:\%(threadName)s:l\%(lineno)d:\%(message)s', `stderr': True, `stderr\_level': `INFO'\}, `Anna': \{\}\}}}
\end{fulllineitems}



\section{Librairie}
\label{profils:librairie}\label{profils:module-lib.conf}\index{lib.conf (module)}\index{Conf (classe dans lib.conf)}

\begin{fulllineitems}
\phantomsection\label{profils:lib.conf.Conf}\pysiglinewithargsret{\strong{class }\code{lib.conf.}\bfcode{Conf}}{\emph{profil}}{}
Classe permettant de gérer les profils de configuration
\begin{quote}\begin{description}
\item[{Paramètres}] \leavevmode
\textbf{profil} (\href{http://docs.python.org/library/string.html\#module-string}{\emph{string}}) -- Profil de configuration

\end{description}\end{quote}
\index{importer\_profil() (méthode lib.conf.Conf)}

\begin{fulllineitems}
\phantomsection\label{profils:lib.conf.Conf.importer_profil}\pysiglinewithargsret{\bfcode{importer\_profil}}{\emph{profil}}{}
Charge un profil de configuration
\begin{quote}\begin{description}
\item[{Paramètres}] \leavevmode
\textbf{profil} (\href{http://docs.python.org/library/string.html\#module-string}{\emph{string}}) -- Profil de configuration

\end{description}\end{quote}

\end{fulllineitems}


\end{fulllineitems}



\chapter{Recherche de chemin}
\label{recherche_chemin::doc}\label{recherche_chemin:recherche-de-chemin}

\section{Graph}
\label{recherche_chemin:graph}\label{recherche_chemin:module-lib.recherche_chemin.graph}\index{lib.recherche\_chemin.graph (module)}\index{Graph (classe dans lib.recherche\_chemin.graph)}

\begin{fulllineitems}
\phantomsection\label{recherche_chemin:lib.recherche_chemin.graph.Graph}\pysigline{\strong{class }\code{lib.recherche\_chemin.graph.}\bfcode{Graph}}
Classe permettant de manipuler des graphs

\end{fulllineitems}



\section{A*}
\label{recherche_chemin:a}\label{recherche_chemin:module-lib.recherche_chemin.astar}\index{lib.recherche\_chemin.astar (module)}\index{Astar (classe dans lib.recherche\_chemin.astar)}

\begin{fulllineitems}
\phantomsection\label{recherche_chemin:lib.recherche_chemin.astar.Astar}\pysigline{\strong{class }\code{lib.recherche\_chemin.astar.}\bfcode{Astar}}
Classe implémentant l'algorithme A* pour la recherche de chemin

\end{fulllineitems}



\section{Theta*}
\label{recherche_chemin:theta}\label{recherche_chemin:module-lib.recherche_chemin.thetastar}\index{lib.recherche\_chemin.thetastar (module)}\index{Thetastar (classe dans lib.recherche\_chemin.thetastar)}

\begin{fulllineitems}
\phantomsection\label{recherche_chemin:lib.recherche_chemin.thetastar.Thetastar}\pysigline{\strong{class }\code{lib.recherche\_chemin.thetastar.}\bfcode{Thetastar}}
Classe implémentant l'algorithme Theta* pour la recherche de chemin

\end{fulllineitems}



\chapter{Robot}
\label{robot:module-lib.robot}\label{robot::doc}\label{robot:robot}\index{lib.robot (module)}\index{Robot (classe dans lib.robot)}

\begin{fulllineitems}
\phantomsection\label{robot:lib.robot.Robot}\pysigline{\strong{class }\code{lib.robot.}\bfcode{Robot}}
Classe qui gère le robot

\end{fulllineitems}



\chapter{Série}
\label{serie:module-lib.serie}\label{serie::doc}\label{serie:serie}\index{lib.serie (module)}\index{Serie (classe dans lib.serie)}

\begin{fulllineitems}
\phantomsection\label{serie:lib.serie.Serie}\pysigline{\strong{class }\code{lib.serie.}\bfcode{Serie}}
Classe de créer une liaison Série
\begin{quote}\begin{description}
\item[{Todo }] \leavevmode
Initialiser la liaison série dans le constructeur

\item[{Todo }] \leavevmode
Fermer proprement la liaison série dans le destructeur

\end{description}\end{quote}

\end{fulllineitems}



\chapter{Simulation des µC}
\label{simul_uc:module-lib.simul_uc}\label{simul_uc::doc}\label{simul_uc:simulation-des-c}\index{lib.simul\_uc (module)}

\chapter{Stratégie}
\label{strategie:module-lib.strategie}\label{strategie::doc}\label{strategie:strategie}\index{lib.strategie (module)}\index{Strategie (classe dans lib.strategie)}

\begin{fulllineitems}
\phantomsection\label{strategie:lib.strategie.Strategie}\pysigline{\strong{class }\code{lib.strategie.}\bfcode{Strategie}}
Classe permettant de construire une stratégie

\end{fulllineitems}



\chapter{Visualisation}
\label{visualisation::doc}\label{visualisation:visualisation}

\section{Série}
\label{visualisation:serie}\label{visualisation:module-lib.visualisation.visu_serie}\index{lib.visualisation.visu\_serie (module)}\index{Visu\_serie (classe dans lib.visualisation.visu\_serie)}

\begin{fulllineitems}
\phantomsection\label{visualisation:lib.visualisation.visu_serie.Visu_serie}\pysigline{\strong{class }\code{lib.visualisation.visu\_serie.}\bfcode{Visu\_serie}}
Classe permettant de visualiser les messages passant par la liaison série

\end{fulllineitems}



\section{Table}
\label{visualisation:table}\label{visualisation:module-lib.visualisation.visu_table}\index{lib.visualisation.visu\_table (module)}\index{Visu\_table (classe dans lib.visualisation.visu\_table)}

\begin{fulllineitems}
\phantomsection\label{visualisation:lib.visualisation.visu_table.Visu_table}\pysigline{\strong{class }\code{lib.visualisation.visu\_table.}\bfcode{Visu\_table}}
Classe permettant de visualiser la table de jeu avec les zones, les éléments de jeu, les robots, ...

\end{fulllineitems}



\section{Threads}
\label{visualisation:module-lib.visualisation.visu_threads}\label{visualisation:threads}\index{lib.visualisation.visu\_threads (module)}\index{Visu\_threads (classe dans lib.visualisation.visu\_threads)}

\begin{fulllineitems}
\phantomsection\label{visualisation:lib.visualisation.visu_threads.Visu_threads}\pysigline{\strong{class }\code{lib.visualisation.visu\_threads.}\bfcode{Visu\_threads}}
Classe permettant de visualiser les différents threads utilisés pour aider à les débuguer

\end{fulllineitems}



\chapter{Indices and tables}
\label{index:indices-and-tables}\begin{itemize}
\item {} 
\emph{genindex}

\item {} 
\emph{modindex}

\item {} 
\emph{search}

\end{itemize}


\renewcommand{\indexname}{Python Module Index}
\begin{theindex}
\def\bigletter#1{{\Large\sffamily#1}\nopagebreak\vspace{1mm}}
\bigletter{l}
\item {\texttt{lib.actionneur}}, \pageref{actionneur:module-lib.actionneur}
\item {\texttt{lib.actions\_in.actionneurs\_in}}, \pageref{actions:module-lib.actions_in.actionneurs_in}
\item {\texttt{lib.actions\_in.asservissement\_in}}, \pageref{actions:module-lib.actions_in.asservissement_in}
\item {\texttt{lib.actions\_in.capteurs\_in}}, \pageref{actions:module-lib.actions_in.capteurs_in}
\item {\texttt{lib.actions\_out.actionneurs\_out}}, \pageref{actions:module-lib.actions_out.actionneurs_out}
\item {\texttt{lib.actions\_out.asservissement\_out}}, \pageref{actions:module-lib.actions_out.asservissement_out}
\item {\texttt{lib.actions\_out.capteurs\_out}}, \pageref{actions:module-lib.actions_out.capteurs_out}
\item {\texttt{lib.asservissement}}, \pageref{asservissement:module-lib.asservissement}
\item {\texttt{lib.capteur}}, \pageref{capteur:module-lib.capteur}
\item {\texttt{lib.carte}}, \pageref{carte:module-lib.carte}
\item {\texttt{lib.conf}}, \pageref{profils:module-lib.conf}
\item {\texttt{lib.decision}}, \pageref{decision:module-lib.decision}
\item {\texttt{lib.elements\_jeu}}, \pageref{elements_jeu:module-lib.elements_jeu}
\item {\texttt{lib.evenement}}, \pageref{evenement:module-lib.evenement}
\item {\texttt{lib.i2c}}, \pageref{i2c:module-lib.i2c}
\item {\texttt{lib.jeu}}, \pageref{jeu:module-lib.jeu}
\item {\texttt{lib.log}}, \pageref{log:module-lib.log}
\item {\texttt{lib.math.point}}, \pageref{math:module-lib.math.point}
\item {\texttt{lib.math.vecteur}}, \pageref{math:module-lib.math.vecteur}
\item {\texttt{lib.peripherique}}, \pageref{peripherique:module-lib.peripherique}
\item {\texttt{lib.recherche\_chemin.astar}}, \pageref{recherche_chemin:module-lib.recherche_chemin.astar}
\item {\texttt{lib.recherche\_chemin.graph}}, \pageref{recherche_chemin:module-lib.recherche_chemin.graph}
\item {\texttt{lib.recherche\_chemin.thetastar}}, \pageref{recherche_chemin:module-lib.recherche_chemin.thetastar}
\item {\texttt{lib.robot}}, \pageref{robot:module-lib.robot}
\item {\texttt{lib.serie}}, \pageref{serie:module-lib.serie}
\item {\texttt{lib.simul\_uc}}, \pageref{simul_uc:module-lib.simul_uc}
\item {\texttt{lib.strategie}}, \pageref{strategie:module-lib.strategie}
\item {\texttt{lib.visualisation.visu\_serie}}, \pageref{visualisation:module-lib.visualisation.visu_serie}
\item {\texttt{lib.visualisation.visu\_table}}, \pageref{visualisation:module-lib.visualisation.visu_table}
\item {\texttt{lib.visualisation.visu\_threads}}, \pageref{visualisation:module-lib.visualisation.visu_threads}
\indexspace
\bigletter{p}
\item {\texttt{profils.develop.constantes}}, \pageref{profils:module-profils.develop.constantes}
\item {\texttt{profils.developSimulUc.constantes}}, \pageref{profils:module-profils.developSimulUc.constantes}
\item {\texttt{profils.prod.constantes}}, \pageref{profils:module-profils.prod.constantes}
\end{theindex}

\renewcommand{\indexname}{Index}
\printindex
\end{document}
